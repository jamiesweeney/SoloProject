\pdfoutput=1

\documentclass{l4proj}

%
% put any packages here
%

\usepackage[]{algorithm2e}
\usepackage{graphicx}
\graphicspath{ {images/} }

\begin{document}
\title{Who's There? Occupancy Prediction with Smart Sensor Boxes}
\author{Jamie Sweeney}
\date{September 26, 2017}
\maketitle

\begin{abstract}
We show how to produce a level 4 project report using latex and pdflatex using the 
style file l4proj.cls
\end{abstract}

\educationalconsent
%
%NOTE: if you include the educationalconsent (above) and your project is graded an A then
%      it may be entered in the CS Hall of Fame
%
\tableofcontents
%==============================================================================


%==============================================================================
\chapter{Introduction}
\pagenumbering{arabic}


\section{Context}
The “Internet of Things”, or IoT is a term used to describe the expanding network of devices of many types,  ranging from supercomputers to household appliances that contain embedded computer systems, making them capable of communication with other IoT devices. These devices aim to bring technology to every corner of our lives, bringing the idea of interconnected smart homes, workplaces, cities and more to reality.

Since the term was first used in 1999\cite{c-iot-term} the IoT has come a long way,  with the number of internet connected devices rapidly increasing\cite{c-devices}, and although nobody seems to agree on global market predictions, many predict massive growth\cite{c-iot-market1}\cite{c-iot-market2}\cite{c-iot-market3}.

One key area of study has been developing occupancy detection systems that would allow IoT devices to be more aware of the occupancy status of their surroundings. This is a crucial step to implementing smart homes, workplaces and campuses that are responsive and can meet their demands at any given time.

Advanced examples of this are already taking concrete form, with the University of Glasgow investing €1bn towards developing a smart campus\cite{c-iot-uog} and the city of Barcelona making major overhauls to many public services in an attempt to bring the idea of a smart city to reality\cite{c-iot-bar}.



\section{Motivations}
The motivations for this project are wide, with smart sensors being applicable in a broad variety of ways. We will discuss the primary motivations for this project, which are understandably more relatable and accessible to myself, in addition to a number of secondary motivations that are considered due to close similarity.

\subsection{Primary Motivation}
To narrow our field of focus, we choose a main motivation to consider when producing the smart sensor boxes. We consider the following scenario: University administration decide they wish to upgrade buildings with ‘smart sensor boxes’ with the overall goal being that students or other staff members can find occupancy information on certain areas remotely. For example library group study areas could be listed online with an indication of their current use, allowing students who plan on visiting to check availability beforehand.
The University of Glasgow library currently provides a similar feature, displaying computer use breakdown by floor, available from monitors within the library.

In addition the system should be built to be easily extendable, and capable of operating as a part of a larger IoT network. This is critical as there are many potential uses for such a system operating on a larger network of IoT devices. One example would be querying the system for occupancy information on a room and adjusting ventilation and heating based on the number of occupants, a effective system would bring potential for large savings in electricity and heating \cite{c-heating} while requiring very little human intervention or overhead costs.

\subsection{Secondary Motivations}
Although we focus on a university implementation of occupancy prediction, we can imagine a similar design could be incorporated into many different areas. Schools, workplaces and public buildings could all benefit from installing smart sensor functionality, allowing not only occupancy information to end users, but also providing more in-depth analysis of overall occupation to any interested administrators.

For example, smart public buildings could allow local councils to analyse the usage of various public services, allowing them to make more informed decisions on funding, maintenance, staffing and other key elements of operation.

To list a few more: Security systems, home-automation, public transport, gyms, dining locations and other areas of leisure. We can see that this problem is applicable to many areas of life and has the potential to shape cities and communites by providing accurate and reliable information that is used network-wide by many parties.


\section{Goals}
In short, the aim of this project is to design and construct low budget ‘smart sensor boxes’, capable of predicting human occupation in a room. To do this we use the popular brand of microprocessors ‘Raspberry Pis’ in addition to a variety of sensors and other techniques.

In addition to the smart sensor boxes, we will design and implement a centralised hub capable of receiving, storing and serving information produced by a series of RPi boxes. This hub should also provide functionality for system-wide maintenance and configuration, to make setup and operation efficient and flexible.

This system should be built to be easily extendable, and capable of operating as part of the IoT network. This is critical as there are many potential uses for such a system operating on a larger network of IoT devices. 

%==============================================================================
\chapter{Background Analysis}
We start by researching and reviewing past studies conducted in similar fields, given that the subject matter is broad and widely applicable, we find many studies available. Many employ sensors and machine learning techniques to achieve meaningful results.


\section{Human Detection}
Sensors are a key element in almost every real-world computing application, especially those concerned with the detection of humans e.g automatic doors. Without them a computing device would have little to no information on its surroundings, and would be extremely ineffective for real-time interactive functionality. For this reason, sensors will be a primary component of the smart boxes.

Background research on previously conducted studies gives us a comprehensive list of applicable sensor, and their specific advantages and disadvantages.

\subsection{Infrared Light}
Infrared light is the electromagnetic radiation emitted by all objects with a temperature above 0°K, giving the feeling of heat. The wavelength of IR light spans from 700 nm to 1 mm, existing just outside of the range of human vision\cite{c-inred}. PIR sensors feature a “sensor face” that can detect a change in the average amount of IR light hitting the surface, the sensor then outputs a signal indicating this change, with a low voltage indicating low change, and a higher voltage indicating a greater change\cite{c-pir}.

They are most commonly used in PIR-based motion detectors such as automatic lighting systems. Generally the purpose of a PIR sensor is to detect the presence of humans at a binary level i.e true or false. However it has been shown that with suitable positioning and utilisation of machine learning techniques, a single PIR sensor can be used to predict the number of occupants in a room within a small range to fair accuracy\cite{c-pir-study}.

There are many problems to overcome when using a PIR sensor for this purpose, most importantly, the positioning of the sensor; PIR sensors are prone to interference from occluding objects i.e blocking one another from the sensor. As a result of this, the most ideal position would be the ceiling of the room, however this may not prove as practical for other aspects such as maintenance\cite{c-pir-study}.

Although it will be difficult to obtain a precise estimation by a PIR sensor alone, with the correct implementation, the sensor could potentially be used alongside other components to achieve a valid predictions.

\subsection{Digital Images}
The field of computer vision first emerged in the 1960s and has been developing steadily ever since with large abouts of research being done for a number of different problems. With endless applications from manufacturing to self-driving vehicles, it’s not surprising that there is a variety of methods available to us, specifically much research has been done on object detection and recognition. 

Primitive solutions to object detection might include matching edges from input imagery to predefined templates or a taking and analysing a histogram of oriented gradients (HOG). 

In recent years, more complex designs have been produced, often in combination with machine learning algorithms and a training dataset which has been shown to be effective in a variety of object detection problem areas. 

Although more complex techniques are tempting, it is important to bear in mind the technical restraints put in place by using a microcontroller; computer vision techniques are notorious for being process intentensive and many require the use of optimised graphical processors to achieve realistic processing times. Fortunately modern digital cameras are extremely advanced and cheap, high quality and compatible digital cameras are readily available, specifically the Raspberry Pi 3 Model B+ features a CSI camera port and a separate camera module. The camera module costs around £20, has an 8-megapixel sensor and is capable of recording in 1080p30 and 720p60\cite{c-picamera} which is most definetely enough for our needs.

To conclude, it seems that computer vision could play a vital role in the final design and is definetely due consideration.

\subsection{Audio}

As humans, audio plays a crutial role in understanding our evironment and communicating with each another, as a result we are able to detect human voices with almost absolute certainty. This close relationship suggests that it is a promising area of study in relation to human detection, i.e we use it sucessfully, therefore machine may be able to do the same. 

We find that this is not quite as promising as previously thought, although a fair amount of research has been done on audio analyis by machine for high level tasks such as human detection, we have yet to see an implementation that can perform anywhere near the level of humans for such tasks. Most sucessful applications of audio analysis have been in speech recognition, for example speech to text functionality which has impoved the accessability of many computer systems and applications.

We find that there a few key problems facing predicting human occupancy using audio, one of them being the fact that noise the environment is constantly disturbing audio detections with noise, from a variety of sources. Much progress has been made over the past few decades concerning the analysis of a single human voice, however little has been made in developing to solutions that would be applicable to mass human detection.

Although audio analyis clearly has the potential to be a major componenet in a smart sensor boxes, our findings suggest that the field needs a lot more development before audio analysis can give meaninful occupancy predictions.

\subsection{Other}
\subsubsection{Luminosity}
Luminosity sensors contain photodiodes which convert light to electric energy, allowing them to effectively measure the lighting level of there surrounding. The typical luminosity range of an office or classroom is between 0 and 500 lux, which is well within the standard sensor range of 0 to 40,000 lux . A commonplace example of an application of these sensors would be street lights which turn on and off. Not much information could be found on cases of luminosity sensors being used for human occupancy prediction, but we find that a cheap and easily available luminosity sensor would be able to provide an indication of occupancy, as a dark room would suggest that there are no occupants. We must also consider the possible interferences that may affect the luminosity of the room, or the reading produced by the sensor. Sunlight shining through a window onto the sensor would greatly increase its reading, and could give an inaccurate reading of the room lighting as a whole.
\subsubsection{Temperature}
Another sign of human occupancy could be room temperature, as occupants naturally heat the air through emission of body heat. However the standard precision of an appropriate temperature sensor is usually 0.5°C, which it too low to be useful in determining exact temperature reading. The temperature change undergone when a group of people enter the room is unlikely be high enough to be properly detected by the sensor and be a guaranteed indication of human occupancy. This is especially true when the room temperature is subject to external factors, for example sun shining through the window will cause the room temperature to increase rapidly. We find that it would be too difficult to separate such an event from temperature rises that indicate human activity, so it follows that temperature sensors would be of little use for this problem.
\subsubsection{CO2 and humidity}
In an unregulated air system, human occupants would also give rise to CO2 and humidity levels through breathing, a higher number of occupants would see that the change in CO2 and humidity would be faster meaning the rate of change could give us a fairly accurate prediction. However for this very reason, building air must be properly regulated to ensure that it is safe for human occupancy, therefore any measurements we took would be expected to be within a fairly small range and would be highly influenced by external factors such as air conditioning, open windows and air flows.


\section{Device Detection}
Although the aim of this system is to detect human occupancy, everyday more and more people are using smartphones or similar electronic wireless devices for almost every aspect of life. It is estimted that ****\% of adults in the UK own a smartphone, and with most of these people using their devices everywhere they go, it is reasonable to say that the number of wireless devices in a specific area would be very highly correlated with the number of human occupants. We must also consider the possability of rooms containing wireless devices that are not relative to the human occupants, for example lab computers may contain Bluetooth functionality that would distory the human to device ratio, however typical workplace machines operate using wired communication, as it is more practical for a stationary machine.

Over the years, smartphone development has brought about faster and more robust wireless protocols and methods. Since many of these protocols are the same ones used by microcontrollers, it is reasonable to think we may also be able to get an estimation of wireless devices nearby using similar techniques to human detection.

Although this system would greatly benifit from being able to detect all devices in the nearby area, obviously this is not technically feasible with any device, since many users would choose to not broadcast their devices and many protocols are not built with device broadcasting in mind, for example communicating on 3G will not let all other nearby 3G devices know the location such as bluetooth communication might.

The only feasible device detection method we can see is using Bluetooth and Bluetooth Low Energy device scanning, which will recieve alerts from nearby devices that have broadcasting enabled. Although not every bluetooth device will broadcast it's location in response to a device scan, the ratio of devices that respond should be relatively constant for similar audiences. 
 

\section{Exitsting products}
During our background analysis we find that similar systems have been implemented before for both research and real-world production use.

"Meshlium", a product of the IoT provider Libelium, attempts to detect WiFi and Bluetooth communications in the nearby area to obtain a number of devices. It is estimated to detect up to 95\% of nearby wireless devices operating on these protocols.

Another IoT solutions provider "Retail Sensing" have developed a CCTV based people counting system which utilises computer vision techniques with CCTV cameras recordings from cities, shopping centres etc. to predict human occupancy by area to up to 98\%.

Unfortunatelty since these systems are a product costing money, the intellectual propery is protected and technical specifications are vague and mostly refer to "our processing algorithms" or something equally uninformative. However we can at least see that these products are feasible and obtain a general overview of how they work which will be a useful reference during the design stage.

%==============================================================================
\chapter{Requirements}

\section{Functional Requirements}

\subsection{Sensor data collection}
The smart sensor boxes should be capable of collecting data from connected sensors for a specified period of time. The connected sensors we should support are:
\begin{enumerate}
  \item Bluetooth device scanning.
  \item Bluetooth Low Energy device scanning.
  \item Camera
\end{enumerate}
However the final system should be easily extendible so that new sensors can be added without too much cost.

\subsection{Reporting}
The smart sensor boxes should be capable of periodically compiling the sensor data into a report and sending it to a centralised storage location.

\subsection{Estimation}
The system should be able to analyse these "Smart Box" reports, to give an estimation of occupancy for each location with reports.

In order to improve the estimation results, there should be some way of manual readings to be submitted and stored. These readings should then be used in combination with related smart sensor reports to tune the estimation process and producess more accurate results.

\subsection{Data serving}
The report data produced should be made available to authorised users and applications.
The estimations produced should be made available to any potential end users and applications.

In addition, authorised users should be able to insert new, delete or amend information on:
\begin{itemize}  
  \item Buildings, floors, rooms.
  \item Smart boxes.
  \item Admin users.
\end{itemize}

\subsection{Data storage}
The system must be capable of storing data about: 
\begin{enumerate}
  \item Buildings, floors, rooms.
\begin{enumerate}
  \item Name.
  \item Description.
\end{enumerate}
  \item Smart boxes.
\begin{enumerate}
  \item Name.
  \item Description.
  \item Location.
  \item Authentication data.
\end{enumerate}
  \item Sensor output.
\begin{enumerate}  
  \item Time.
  \item Sensor reading.
\end{enumerate}
  \item Smart box reports.
\begin{enumerate} 
  \item Time.
  \item Sensor reading.
\end{enumerate}
  \item Estimations.
\begin{enumerate} 
  \item Time.
  \item Location.
  \item Estimate.
\end{enumerate}
  \item Manual readings
\begin{enumerate} 
  \item Time frame.
  \item Location.
  \item Reading.
\end{enumerate}
  \item Admin users.
\begin{enumerate} 
  \item Username.
  \item Hashed password.
\end{enumerate}
\end{enumerate}

\section{Non-functional Requirements}
\subsection{Sensor data collection}
\begin{itemize}
  \item Ideally each smart sensor box should be as cheap as possible while providing adequte data, in order for the system to be scaled up without large overhead costs.
  \item For each sensor method, if the data can be collected continuously i.e results can be streamed in real time, then it should be done so with as little downtime as possible. If the data cannot be collected continuously, meaning they require some amount of time to compute results before obtaining a new reading, then we should perform as many iterations as possible to ensure that results are up to date.
  \item The data collection should be able to run for as long as possible without human intervention to reduce the amount of maintainence required.
\end{itemize}

\subsection{Reporting}
\begin{itemize}	
  \item To ensure that occupancy reports and predictions are not outdated, RPis should be able to send reports at a rate of atleast one report per minute. The reports should also stick to the specified rate with fair accuracy and not deviate too far from a steady report rate.
  \item To prevent irregularities in the sensor data affecting reports, the reports should use some form of smoothing mechanism, which would cause any sudden changes in sensor readings to have a smaller effect on the overall estimate.
  \item Report structure should allow for partial reports from sensor boxes not equipped with all sensors, and also allow for easy extensibility to allow for new sensor additions.
  \item Report structure should take a common format such as JSON.
\end{itemize}

\subsection{Estimation}
\begin{itemize}	
  \item The estimates should be as accurate as possible.
  \item Estimates should be produced at a reasonable enough rate to ensure that they are not out of date.
\end{itemize}

\subsection{Data serving}
All data served to users should be:
\begin{itemize}	
  \item In a user-friendly and informative format.
  \item As close to real time as possible.
  \item Data 

\end{itemize}

\subsection{Data storage}
Any data stored should:
\begin{itemize}	
  \item Be implemented in an efficient and correct manner, without unnecessary dependencies between data.
  \item Easily scaleable without any major modifications.
  \item Accessible as quickly as possible and with no down-time.
  \item Secure from malicious attack or unauthorised access.
\end{itemize}



%==============================================================================
\chapter{Design}
\section{High Level Summary}

We start by splitting the requirements to three seperate components with individual functions.
\begin{enumerate}
  \item Smart Sensor Boxes:
	Collects data from the various sensors, also generate reports detailing RPi identification, allong with smoothed sensor data. Reports are then sent to the HTTP server for analysis. 
  \item Centralised Storage:
	A database that allows for the storing and querying of data must contain information on building, RPis, estimates, reports, users etc.
  \item HTTP Server:
	A simple HTTP application that can receive sensor report data from the smart sensor boxes. The sever should also be able to perform estimations for locations based on reports and also feature a web application that serves this data to the users. 
\end{enumerate}
Figure \ref{fig:overalldiag} Shows a simple high level diagram of the system design.
\begin{figure}
\centering
\includegraphics[width=\textwidth]{overalldiag}
\caption{High level design of system.}
\label{fig:overalldiag}
\end{figure}

\section{Components}
\subsection{Smart Sensor Boxes}
The smart sensor box logic consist of a main reporter algorithm which initialises scanning for each detection method and monitors the output. Every time a report is required, the algorithm takes the outputs from each output queue, and collates them to a report which is sent to the HTTP server for furthur use.  Algorithm \ref{RPA} shows the pseudocode for this algorithm.

The detection methods supported are:
\begin{enumerate}
  \item Bluetooth / BTLE device scanning -  Continuously scans for devices and adds any discoveries to a output map, scans are done for a specified time before restarting to ensure that devices do not stop responding the the scan signal. Algorithm \ref{BTS} details the algorithms used.
  \item Camera and object detection -  Runs on a loop taking images and feeding them into an object detection algorithm which predicts objects and their certainty. The number of people is extracted from the detection output and smoothed with previous results using an average of the previous value and the new prediction. Algorithm \ref{CMS} details the algorithms used.
\end{enumerate}

\begin{algorithm}
\DontPrintSemicolon
\nl $\textbf{void}~scanBluetooth(\textbf{Int}~T_{cycle}, \textbf{Int}~T_{timeout}, \textbf{Int}~T_{decay}, \textbf{Map}~output)$ \;
\nl \Begin{
\nl \While {$T_{timeout} > T_{now}$}{
\nl $scan(T_{cycle}, T_{decay})$\;
\nl \While {$scanning$}{
\nl $discovery \gets new\textunderscore discovery$\;
\nl $handleDiscovery(T_{decay}, discovery, output)$\;
}
}
}
\;
\nl $\textbf{void}~handleDiscovery(\textbf{Int}~T_{decay}, \textbf{Map}~discovery, \textbf{Map}~output)$ \;
\nl \Begin{
\nl $addr \gets discovery['address']$\;
\nl $rssi \gets discovery['rssi']$\;
\nl $time \gets discovery['time']$\;
\nl \eIf {$address\ in\ output$}{
\nl $old \gets output[addr]$\;
\nl \eIf {$old['time'] + T_{decay} < T_{now} $}{
\nl $new \gets \{'rssi': (rssi + old['rssi'])/2, 'time': time\}$\;
\nl $output[addr] \gets new$\;
\nl }{
\nl $new \gets \{'rssi': rssi, 'time': time\}$\;
\nl $output[addr] \gets new$\;
}
}{
\nl $new \gets \{'rssi': rssi, 'time': time\}$\;
\nl $output[addr] \gets new$\;
}
}
\;
\caption{Pseudocode for Bluetooth and Bluetooth Low Energy device scanners.}
\label{BTS}
\end{algorithm}

\begin{algorithm}
\DontPrintSemicolon
\nl $\textbf{void}~scanCamera(\textbf{Int}~T_{cycle}, \textbf{Int}~T_{timeout}, \textbf{Int}~output)$ \;
\nl \Begin{
\nl \While {$T_{timeout} > T_{now}$}{
\nl $img \gets takeImage()$\;
\nl $objs \gets objectDetection(img)$\;
\nl $hum \gets numHuman(objs)$\;
\nl $output \gets (output + hum)/2$\;
\nl $wait(T_{cycle})$\;
}
}
\;
\caption{Pseudocode for the camera detection algorithm.}
\label{CMS}
\end{algorithm}

\begin{algorithm}
\DontPrintSemicolon
\nl $\textbf{void}~reporter(\textbf{Int}~T_{cycle}, \textbf{Int}~T_{timeout})$ \;
\nl \Begin{

\nl $T_{decay}, T_{ccycle}, T_{bcycle} \gets getConfig()$\;
\nl $auth \gets getAuthData()$\;


\nl $bt\textunderscore output \gets new\ \textbf{Map}$\;
\nl $cm\textunderscore output \gets 0$\;

\nl $scanBluetooth(T_{bcycle}, T_{timeout}, T_{decay},bt\textunderscore output)$\;
\nl $scanCamera(T_{ccycle},T_{timeout}, cm\textunderscore output)$\;

\nl \While {$T_{timeout} > T_{now}$}{

\nl $devices \gets bt\textunderscore output$\;
\nl $people \gets cm\textunderscore output$\;
\nl $report \gets \{ 'devices':devices, 'people':people, 'time': T_{now}, 'auth': auth\}$\;
\nl $sendReport(report)$\;
\nl $wait(T_{cycle})$\;
}
}
\;

\caption{Pseudocode for the reporter algorithm.}
\label{RPA}
\end{algorithm}


\subsection{Centralised Storage}
\begin{itemize}	
  \item Show ER diagram.
Figure \ref{fig:ERdiagram} Shows an ER diagram of database design.
\begin{figure}
\centering
\includegraphics[width=\textwidth]{ERdiagram}
\caption{ER diagram of database design.}
\label{fig:ERdiagram}
\end{figure}




  \item Discuss relations.
  \item storing passwords as hash
\end{itemize}

\subsection{HTTP Server}





%==============================================================================
\chapter{Implementation}


\section{High Level Summary}
\section{Components}
\subsection{Smart Sensor Boxes}
The smart sensor boxes were implemented using the popular brand of microprocessors “Raspberry Pi”, the RPis were a great choice for a smart sensor box such as this due to the fact that it is compatible with a wide variety of sensors and have relatively powerful computation ability. Specifically we use the most recent version, “Raspberry Pi 3 Model B” which features inbuilt WiFi and Bluetooth adapters, as well as a dedicated RPi camera port.

The RPi camera port allows us to take still images at a resolution of 5 megapixels, using the Python module “picamera”.
Once captured, still images are then passed as argument through a terminal command to an implementation of the object detection algorithm “You Only Look Once”. An original implementation “darknet” was chosen, however the RPi inbuilt GPU is not very powerful and was taking 400 seconds per still image. To mediate this we switched to a darknet version that was optimised of ARM processors, which brought the total time taken down to 40 seconds. This would have been sufficient, however it was found that by using a less configuration files package,  “tiny-YOLO”, we could improve the time further to 4 seconds, with relatively little difference in accuracy.

Since the RPi 3B also has an inbuilt bluetooth adapter, we can use the Python module “bluez” to constantly perform a scan for bluetooth devices that are broadcasting nearby.
Similarly, we use the module “bluepy” to scan for Bluetooth Low Energy devices.

This functionality is implemented as 3 separate python scripts that can be called as subprocess from the reporter process. We pass in an output queue to each script on execution to allow the main process to read any devices and YOLO predictions.

\subsection{Centralised Storage}
We implement our centralised storage using a Google Cloud SQL instance which hosts a MySQL 5.7 server containing our databases. The users table is implemented as one database named "users", all the other tables are contained within another database named "reports". To access the SQL instance, we add the IP address of the connecting device to the authorised networks list. To ensure that all data transmitted to and from the SQL instance is encrypted end to end, we can enfore SSL connections and generate client certificates that can be used by client devices (servers, development environments etc.).

For added protection against database corruption or loss of data, we can enable automatic backups through the Google Cloud console which would allow us to restore lost data from previous copies of the database.
 

\subsection{HTTP Server}

In order to quickly develop and deploy a web server, we use the Python microframework “Flask” which requires very little setup or configuration. 

To access the central storage instance, we create a user for the server and add the IP to authorised network list. Then we can use the Python module "MySQLdb" which provides an interface for python to call mysql. In order to keep communication secure we must also store the SSL certificates with the application so they are accessible by MySQLdb.

The server must also allow admin functionality, this requires us to use some method of authentication for any admin users. We use the “Flask-login” builtin functionality to implement a simple username/password system with new user registration restricted to admin users. Endpoints that require authentication can now simply be wrapped with "@login\_required" which will ensure that any responses not authenticated will be require to provide login details. After logging in, authentication data is stored in session cookies that are safe from tamper.

Although suitable for development of web applications, Flask alone is not sufficient for deploying applications to a production environment due to the fact that it scales poorly. To use our Flask application in a production environment, we need to use "mod\_wsgi" package which will host our application, making it suitable for realistic usage.

The production environment is installed on a Google Cloud compute engine which serves the webpage, to serve this to the public we obtain the domain "peoplecount.cf" and route DNS for that domain to the static IP address of the instance. Web pages are served through the url scheme "peoplecount.cf/webapp/..." and all API endpoints follow the scheme  "peoplecount.cf/api/...". An example of both of these would be "peoplecount.cf/webapp/home" and "peoplecount.cf/api/buildings/get-all".

Since the data our application will be sending and receiving must be secure, we must ensure that our application serves and receives data through HTTP over TLS (HTTPS), which will ensure that the data is end-to-end encrypted and safe from interception.

To make sure that HTTPS is enforced, we set up a redirect from any HTTP requests to the HTTPS equivalent.



%==============================================================================
\chapter{Evaluation}



%==============================================================================
\chapter{Conclusion}
\section{Summary}
\section{Future Work}
\section{Reflection}


%%%%%%%%%%%%%%%%
%              %
%  APPENDICES  %
%              %
%%%%%%%%%%%%%%%%
\begin{appendices}

\chapter{Title of appendix}
Put some data in here:
\begin{verbatim}
      > blahblahblah
\end{verbatim}

\end{appendices}

%%%%%%%%%%%%%%%%%%%%
%   BIBLIOGRAPHY   %
%%%%%%%%%%%%%%%%%%%%

\bibliographystyle{plain}
\bibliography{bib}

\end{document}
